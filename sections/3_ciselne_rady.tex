\section{Číselné řady}

\subsection*{Definice číselné řady}

Formální výraz tvaru

\[ \sum_{k=n_0}^\infty a_k = a_{n_0} + a_{n_0 + 1} + a_{n_0 + 2} + \cdots, \]

\noindent kde $(a_k)_{k=n_0}^\infty$ je zadaná číselná posloupnost, nazýváme \textbf{číselnou
    řadou}. Pokud je \textbf{posloupnost částečných součtů}
$(s_n)_{n=n_0}^\infty$ definovaná předpisem

\[ s_n \equiv \sum_{k=n_0}^n a_k, \quad n\in\mathbb{N}_0, n \geq n_0, \]

\noindent konvergentní, nazýváme příslušnou řadu také konvergentní. V opačném
případě o ní mluvíme jako o divergentní číselné řadě.
Součtem konvergentní řady $\sum_{k=n_0}^\infty a_k$ nazýváme hodnotu
limity $\displaystyle\lim_{n\to\infty} s_n$.

\subsection*{Nutná podmínka konvergence}

Pokud řada $\sum_{k=0}^\infty a_k$ konverguje, potom pro limitu jejích sčítanců
platí $\displaystyle \lim_{k\to\infty} a_k = 0$.

\subsection*{Bolzanovo--Cauchyovo kritérium pro řady}

Řada $\displaystyle\sum_{k=0}^\infty a_k$ konverguje právě tehdy, když pro každé $\epsilon > 0$ existuje $n_0 \in \mathbb{R}$ tak, že pro každé přirozené $n \geq n_0$ a $p \in \mathbb{N}$ platí

\[ |a_n + a_{n+1} + \cdots + a_{n+p}| < \epsilon. \]

\subsection*{Absolutní konvergence}

Číselnou řadu $\sum_{k=0}^\infty a_k$ nazýváme \textbf{absolutně konvergentní}, pokud číselná řada $\sum_{k=0}^\infty |a_k|$ konverguje.

\subsection*{O vztahu absolutní konvergence a konvergence}

Pokud řada absolutně konverguje, \textbf{potom} tato řada konverguje.

\subsection*{Leibnizovo kritérium}

Buď $(a_k)_{k=0}^\infty$ monotónní posloupnost konvergující k nule. Potom je
řada

\[ \sum_{k=0}^\infty (-1)^k a_k \]

\noindent konvergentní.

\subsection*{Srovnávací kritérium}

Buďte $\sum_{k=0}^\infty a_k$ a $\sum_{k=0}^\infty b_k$ číselné řady. Potom
platí následující dvě tvrzení.

\begin{enumerate}
    \item Nechť existuje $k_0 \in \mathbb{N}$ takové, že pro každé $k\in\mathbb{N}$ větší
          než $k_0$ platí nerovnosti $0 \leq |a_k| \leq b_k$ a nechť řada
          $\sum_{k=0}^\infty b_k$ konverguje. Potom řada $\sum_{k=0}^\infty a_k$
          absolutně konverguje.
    \item Nechť existuje $k_0 \in \mathbb{N}$ takové, že pro každé $k\in\mathbb{N}$ větší
          nebo rovno než $k_0$ platí nerovnosti $0 \leq a_k \leq b_k$ a
          $\sum_{k=0}^\infty a_k$ diverguje. Potom i řada $\sum_{k=0}^\infty b_k$
          diverguje.
\end{enumerate}

\subsection*{D'Alembertovo kritérium}

Nechť $a_k > 0$ pro každé $k\in\mathbb{N}_0$. Pokud

\[ \lim_{k\to\infty} \frac{a_{k+1}}{a_k} > 1, \]

\noindent potom řada $\displaystyle\sum_{k=0}^\infty a_k$ diverguje. Pokud ovšem

\[ \lim_{k\to\infty} \frac{a_{k+1}}{a_k} < 1, \]

\noindent potom řada $\displaystyle\sum_{k=0}^\infty a_k$ konverguje.

\subsection*{O odhadu posloupnosti částečných součtů}

Nechť $f$ je spojitá funkce na $\langle 1,+\infty)$ a $n\in\mathbb{N}$. Je-li
$f$ klesající, pak platí

\[ f(n) + \int_1^n f(x) \,\mathrm{d} x \leq \sum_{k=1}^n f(k) \leq f(1) + \int_1^n f(x) \,\mathrm{d}x. \]

\noindent Je-li $f$ rostoucí, pak platí

\[ f(1) + \int_1^n f(x) \,\mathrm{d} x \leq \sum_{k=1}^n f(k) \leq f(n) + \int_1^n f(x) \,\mathrm{d}x. \]

\subsection*{Integrální kritérium}

Buď $\displaystyle\sum_{n=1}^\infty a_n$ číselná řada s kladnými členy taková,
že existuje spojitá a monotónní funkce definovaná na $\langle 1,+\infty)$
taková, že $f(n) = a_n$ pro každé $n$. Potom

\begin{itemize}
    \item Pokud (zobecněný Riemannův) integrál $\displaystyle\int_1^\infty
              f(x)\,\mathrm{d}x$ konverguje, pak číselná řada $\displaystyle\sum_{n=1}^\infty
              a_n$ konverguje.
    \item Pokud integrál $\displaystyle\int_1^\infty f(x)\,\mathrm{d}x$ diverguje, pak
          číselná řada $\displaystyle\sum_{n=1}^\infty a_n$ diverguje.
\end{itemize}

\subsection*{Exponenciální funkce a Eulerovo číslo}

Zobrazení, které každému $x\in\mathbb{R}$ přiřazuje součet konvergentní řady

\[ \sum_{k=0}^\infty \frac{x^k}{k!}, \]

\noindent nazýváme \textbf{exponenciální funkcí}. Její funkční hodnotu v bodě $x$
značíme symbolem $e^x$. Platí tedy

\[ e^x \equiv \sum_{k=0}^\infty \frac{x^k}{k!}, \quad x\in\mathbb{R}. \]

\subsection*{Základní vlastnosti exponenciální funkce}

Exponenciální funkce oplývá následujícími vlastnostmi:

\begin{enumerate}
    \item $e^0 = 1$,
    \item pro všechna $x,y\in\mathbb{R}$ platí $e^{x+y} = e^x e^y$,
    \item pro všechna $x\in\mathbb{R}$ platí $e^x > 0$ a dále $e^{-x} = \frac{1}{e^x}$,
    \item exponenciála je ostře rostoucí funkce, pro všechna $x,y \in \mathbb{R}$
          splňující nerovnost $x < y$ platí nerovnost $e^x < e^y$.
\end{enumerate}

\subsection*{Eulerovo číslo}
\textbf{Eulerovo číslo} definujeme pomocí exponenciální funkce předpisem

\[ e \equiv e^1 = \sum_{k=0}^\infty \frac{1}{k!}. \]

\noindent \textbf{Eulerovo číslo} je iracionální.

\subsection*{Přirozený logaritmus}

Existuje tedy inverzní funkce k exponenciále, která je také ostře rostoucí a
zobrazuje $(0,+\infty)$ na $\mathbb{R}$. Tuto funkci nazýváme
\textbf{přirozeným logaritmem} a značíme symbolem $\ln$.

\subsection*{Vlastnosti přirozeného logaritmu}

Přirozený logaritmus $\ln$ oplývá následujícími vlastnostmi:

\begin{enumerate}
    \item pro každé $x\in\mathbb{R}$ platí $\ln e^x = x$ a pro každé $x\in(0,+\infty)$
          platí
    \item $e^{\ln x} = x$,
    \item $\ln e = 1$ a $\ln 1 = 0$,
    \item pro $x,y \in (0,+\infty)$ platí $\ln(xy) = \ln x + \ln y$.
\end{enumerate}

\subsection*{Obecná mocnina}

Pro $a\in(0,+\infty)$ a $x\in\mathbb{R}$ definujeme

\[ a^x \equiv e^{x\ln a}. \]

\subsection*{Vlastnosti obecné mocniny}

Pro $a,b > 0$ platí

\begin{enumerate}
    \item $a^{x+y} = a^x a^y$,
    \item $\big(a^x\big)^y = a^{xy}$,
    \item $(ab)^x = a^x b^x$.
\end{enumerate}

\noindent pro libovolná $x,y\in\mathbb{R}$.

\pagebreak
\section{Taylorovy polynomy}

\subsection*{Polynom}

Reálnou funkci reálné proměnné $p: \mathbb{R} \to \mathbb{R}$ nazveme
\textbf{polynomem}, právě když existuje nezáporné celé číslo $n\in\mathbb{N}_0$
a reálná čísla $a_0, \ldots, a_n \in \mathbb{R}$ taková, že rovnost

\[ p(x) = \sum_{k=0}^n a_k x^k \]

\noindent platí pro všechna reálná $x\in\mathbb{R}$.

\subsection*{Taylorův polynom}

Nechť reálná funkce reálné proměnné $f$ má v bodě $a\in\mathbb{R}$ konečnou
$n$-tou derivaci. Polynom

\[ T_{n,a}(x) = \sum_{k=0}^n \frac{f^{(k)}(a)}{k!} (x-a)^k \]

\noindent nazýváme \textbf{$n$-tým Taylorovým polynomem funkce $f$ v bodě $a$}

\subsection*{Věta o vlastnostech Taylorova polynomu}

Nechť reálná funkce reálné proměnné $f$ má v bodě $a\in\mathbb{R}$ konečnou
$n$-tou derivaci. Potom Taylorův polynom $T_{n,a}$ existuje a je to jediný
polynom stupně nejvýše $n$ takový, že

\[ T_{n,a}^{(k)}(a) = f^{(k)}(a) \ \text{pro každé} \ k=0,1,\ldots,n. \]

\subsection*{Taylorův vzorec a Taylorův zbytek}

Nechť funkce $f$ má v bodě $a$ konečnou $n$-tou derivaci. Pro všechna přípustná
$x$ položme $R_{n,a}(x) \equiv f(x) - T_{n,a}(x)$. Potom vztah

\[ f(x) = T_{n,a}(x) + R_{n,a}(x) \]

\noindent nazýváme \textbf{Taylorovým vzorcem} a $R_{n,a}$ nazýváme \textbf{$n$-tým zbytkem} v Taylorově vzorci.

\subsection*{Věta o zbytku v Taylorově vzorci}

Nechť funkce $f$ má v jistém okolí $U_a$ bodu $a$ spojitou $n$-tou derivaci.
Pak pro zbytek v Taylorově vzorci platí

\[ \lim_{x\to a} \frac{R_{n,a}(x)}{(x-a)^n} = 0. \]

\subsection*{Věta o nejlepší aproximaci}

Nechť funkce $f$ má v jistém okolí bodu $0$ konečnou $n$-tou derivaci a nechť
$Q$ je polynom stupně nejvýše $n$, různý od Taylorova polynomu $T_n$ funkce $f$
v bodě $0$. Potom existuje okolí $U_0$ bodu $0$ takové, že

\[ |f(x) - T_n(x)| < |f(x) - Q(x)| \quad \textrm{pro každé} \ x\in U_0 \smallsetminus \{0\}. \]

\subsection*{Taylorova věta}

Nechť existuje okolí $U_a$ bodu $a$ takové, že funkce $f$ v něm má konečnou
$(n+1)$ -ní derivaci. Pak zbytek v Taylorově vzorci $f(x) = T_{n,a}(x) +
    R_{n,a}(x)$ lze pro každé $x\in U_a$ zapsat ve tvaru

\[ R_{n,a}(x) = \frac{f^{(n+1)}(\xi)}{(n+1)!} (x-a)^{n+1}, \]

\noindent kde číslo $\xi$ závisí na $x$ a $n$ a leží uvnitř intervalu s krajními body $x$
a $a$. Tento tvar zbytku nazýváme \textbf{Lagrangeův}.

\subsection*{Mocninná řada}

Nechť je dána posloupnost $(a_k)_{k=0}^\infty$ a číslo $c\in\mathbb{R}$.
Číselnou řadu

\[ \sum_{k=0}^\infty a_k (x-c)^k, \]

\noindent závisející na reálném parametru $x$, nazýváme \textbf{mocninnou řadou se
    středem v bodě $c$}.

\subsection*{Taylorova řada}

Nechť reálná funkce reálné proměnné $f$ má v bodě $c \in \mathbb{R}$ konečné
derivace všech řádů. Mocninnou řadu

\[ \sum_{k=0}^\infty \frac{f^{(k)}(c)}{k!} (x-c)^k \]

\noindent potom nazýváme \textbf{Taylorovou řadou funkce $f$ v bodě $c$}.

\subsection*{Věta o poloměru konvergence}

Pokud existuje limita

\[ L := \lim_{k\to\infty} \bigg|\frac{a_{k+1}}{a_k}\bigg|, \]

\noindent potom klademe

\[ R := \begin{cases} \frac{1}{L}, & L > 0, \\ +\infty, & L = 0, \\ 0, & L = +\infty \end{cases} \]

\noindent a tvrdíme, že mocninná řada

\[ \sum_{k = 0}^\infty a_k (x-c)^k  \]

\noindent \textbf{konverguje absolutně} pro $x\in(c-R,c+R)$ a diverguje pro $|c-x| > R$.

\subsection*{Cauchyho--Hadamardova věta}

Ke každé mocninné řadě tvaru

\[ \sum_{k=0}^\infty a_k x^k. \]

\noindent existuje $R\in\langle 0,+\infty\rangle$ takové, že řada absolutně konverguje
pro $|x| < R$ a diverguje pro $|x| > R$.

\pagebreak

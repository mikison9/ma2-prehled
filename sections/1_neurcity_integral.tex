\section{Neurčitý integrál}

\subsection*{Primitivní funkce}

Nechť $f$ je funkce definovaná na intervalu $(a, b)$, kde $-\infty \leq a \leq
    b \leq \infty$. Funkci $F$ splňující

\[ F'(x) = f(x), \forall  x \in (a, b)\]

\noindent nazýváme \textbf{primitivní funkcí} k funkci $f$ na intervalu $(a, b)$.

\subsection*{Věta o jednoznačnosti primitivní funkce}

Nechť $F$ je primitivní funkcí k funkci $f$ na intervalu $(a, b)$. Pak $G$ je
primitivní funkcí k funkci $f$ na intervalu $(a, b)$ právě tehdy, když existuje
konstanta $C \in \mathds{R}$ taková, že

\[ G(x) = F(x) + C, \forall x \in (a, b)\]

\subsection*{Neurčitý integrál}

Nechť k funkci $f$ existuje primitivní funkce na intervalu $(a, b)$. Množinu
všech primitivních funkcí k funkci $f$ na $(a, b)$ nazýváme \textbf{neurčitým
    integrálem} a značíme jej $\int f$ nebo $\int f(x)dx$

\subsection*{Postačující podmínka pro existenci primitivní funkce}

Nechť funkce $f$ je spojitá na intervalu $(a, b)$. Pak má funkce $f$ na tomto
intervalu primitivní funkci.

\subsection*{Linearita primitivní funkce}

Nechť $F$, resp. $G$, je primitivní funkce k funkci $f$, resp. $g$, na
intervalu $(a, b)$ a nechť $\alpha \in \mathds{R}$. Pak

\begin{itemize}
    \item $F + G$ je primitivní funkcí k funkci $f + g$ na intervalu $(a, b)$.
    \item $\alpha F$ je primitivní funkcí k funkci $\alpha f$ na intervalu $(a, b)$.
\end{itemize}

\subsection*{Integrace per partes}

Nechť funkce $f$ je diferencovatelná na intervalu $(a, b)$ a $G$ je primitivní
funkce k funkci $g$ na intervalu $(a, b)$ a konečně nechť existuje primitivní
funkce k funkci $f'G$. Potom existuje primitivní funkce k funkci $fg$ a platí

\[ \int fg = fG - \int f'G\]

\subsection*{První věta o substituci}

Nechť pro funkce $f$ a $\varphi$ platí

\begin{enumerate}
    \item $f$ má primitivní funkci $F$ na intervalu $(a, b)$,
    \item $\varphi$ je na intervalu $(\alpha, \beta)$ diferencovatelná,
    \item $\varphi((\alpha, \beta) \subset (a, b))$.
\end{enumerate}

\noindent Pak funkce $f(\varphi(x))\cdot \varphi'(x)$ má primitivní funkci na intervalu $(\alpha, \beta)$ a platí

\[ \int f(\varphi(x)) \cdot \varphi'(x) dx = F(\varphi(x)) + C \]

\noindent kde $C$ je integrační konstanta.

\subsection*{Druhá věta o substituci}

Nechť $f$ je definována na intervalu $(a, b)$ a nechť $\varphi$ je bijekce
intervalu $(\alpha, \beta)$ na $(a, b)$ s nenulovou konečnou derivací. Pak
platí

\[ \int f(\varphi(t))\varphi'(x)dt = G(t) + C \Rightarrow \int f(x)dx = G(\varphi^{-1}(x)) + C \]

\noindent kde $C$ je integrační konstanta.

\pagebreak

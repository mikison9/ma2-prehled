\section{Určitý integrál}

\subsection*{Dělení intervalu}
Buď dán interval $\langle a, b \rangle$. Konečnou množinu

\[ \sigma = \{x_0, x_1, \ldots, x_n\} \]

\noindent takovou, že

\[ a = x_0 < x_1 < \cdots < x_n = b \]

\noindent nazýváme \textbf{dělením intervalu} $\langle a,b \rangle$. Bodům $x_k$,
$k=1,2,\ldots,n-1$, říkáme \textbf{dělící body intervalu} $\langle a,b
    \rangle$. Intervalu $\langle x_{k-1},x_k \rangle$ říkáme \textbf{částečný
    interval} intervalu $\langle a,b \rangle$ při dělení $\sigma$.

\noindent Číslo:

\[ \nu(\sigma) \equiv \max \{ \Delta_k \mid k = 1,2,\ldots,n \},
    \quad \text{kde} \quad \Delta_k \equiv x_k - x_{k-1}, \ k = 1,2,\ldots,n, \]

\noindent nazýváme \textbf{normou dělení} $\sigma$.

\subsection*{Ekvidistantní dělení}

Pro interval $\langle a,b \rangle$ a $n\in\mathbb{N}$ položme $\Delta \equiv
    \frac{b-a}{n}$ a

\[x_i \equiv a + i\cdot\Delta, \quad i = 0,1,\ldots,n.\]

\noindent Tedy

\[ \sigma = \big\{ a, \, a + \Delta, \, a + 2 \Delta, \ldots, \, a + (n-1)\Delta, \, b \big\}. \]

\subsection*{Dolní a horní součet funkce při dělení $\sigma$}

\noindent Buďte funkce $f$ definovaná a omezená na intervalu $J =\langle a,b \rangle$ a $\sigma = \{x_0,x_1,\ldots,x_n\}$ dělení intervalu $J$. Součty

\[S(\sigma,f) \equiv \sum_{i=1}^n \Delta_i \sup_{\langle x_{i-1},x_i\rangle} f,\]
\[ s(\sigma,f) \equiv \sum_{i=1}^n \Delta_i \inf_{\langle x_{i-1},x_i\rangle} f \]

\noindent nazýváme \textbf{horním součtem funkce} a \textbf{dolním součtem funkce} $f$ při dělení $\sigma$.

\subsection*{Dolní a horní integrál}

Pro funkci $f$ definovanou a omezenou na uzavřeném intervalu $J=\langle a,b
    \rangle$ pomocí \textbf{dolních a horních součtů} definujeme čísla

\[ \overline{\int_a^b} f(x) \,dx \equiv \inf\{ S(\sigma, f) \mid \sigma \ \text{dělení} \ J \}, \]

\[ \underline{\int_a^b} f(x) \,dx \equiv \sup\{ s(\sigma, f) \mid \sigma \ \text{dělení} \ J \}. \]

\noindent a nazýváme je \textbf{horním integrálem}, resp. \textbf{dolním integrálem}, funkce $f$ na intervalu $J$.

\subsection*{Riemannův určitý integrál}

Mějme funkci $f$ definovanou a omezenou na uzavřeném intervalu $J$. Pokud pro
její dolní a horní integrál na intervalu $J$ platí

\[ \overline{\int_a^b} f(x)\,dx = \underline{\int_a^b} f(x) \,dx \in \mathbb{R}, \]

\noindent pak jejich společnou hodnotu nazýváme \textbf{Riemannovým integrálem} funkce $f$ na intervalu $J$ a toto číslo značíme symboly

\[ \int_a^b f, \quad \text{případně} \quad \int_a^b f(x)\,dx. \]

\subsection*{Normální posloupnost dělení}

Posloupnost dělení $\sigma_n$ nazveme \textbf{normální}, pokud pro její normy
platí

\[ \lim_{n\to\infty} \nu(\sigma_n) = 0. \]

\subsection*{Postačující podmínka pro existenci Riemannova integrálu}

Buď $f$ spojitá funkce na intervalu $\langle a,b \rangle$. Potom existuje její
Riemannův integrál na intervalu $\langle a,b\rangle$.

Pokud je navíc $(\sigma_n)$ normální posloupnost dělení intervalu $\langle a,b
    \rangle$, potom limity

\[ \lim_{n\to\infty} s(\sigma_n, f) \quad \text{a} \quad \lim_{n\to\infty} S(\sigma_n,f) \]

\noindent existují, a jsou rovny Riemannově integrálu funkce $f$ na intervalu $\langle a,b \rangle$.
\subsection*{Integrální součet}

Pro funkci $f$ spojitou na uzavřeném intervalu $\langle a,b \rangle$ a dělení
$\sigma = \{x_0,\,x_1,\ldots,\,x_n\}$, kde $x_0 = a$ a $x_n = b$, tohoto
intervalu definujeme \textbf{integrální součet} funkce $f$ při dělení $\sigma$

\[ \mathcal{J}(\sigma,f) = \sum_{i=1}^n f(\alpha_i) \Delta_i, \]

\noindent kde $\alpha_i$ patří do intervalu $\langle x_{i-1}, x_i \rangle$,
$i=1,2,\ldots,n$.

\subsection*{Aditivita integrálu}

Nechť $f$ a $g$ jsou spojité funkce na intervalu $\langle a,b \rangle$. Potom
pro Riemannův integrál funkce $f+g$, která je také automaticky spojitá na
intervalu $\langle a,b \rangle$, platí

\[ \int_a^b (f+g)(x)\,dx = \int_a^b f(x)\,dx + \int_a^b g(x)\,dx. \]

\subsection*{Multiplikativita integrálu}

Nechť $f$ je spojitá na intervalu $\langle a,b \rangle$ a $c\in\mathbb{R}$ je
konstanta. Potom pro Riemannův integrál funkce $cf$ platí

\[ \int_a^b (cf)(x)\,dx = c \int_a^b f(x)\,dx. \]

\subsection*{Aditivita integrálu v mezích}

Riemannův integrál funkce $f$ na intervalu $\langle a, b \rangle$ existuje,
právě když pro každé $c \in (a,b)$ existují Riemannovy integrály funkce $f$ na
intervalech $\langle a, c \rangle$ a $\langle c, b \rangle$. V takovém případě
navíc platí

\[ \int_a^b f(x)\,dx  = \int_a^c f(x)\,dx + \int_c^b f(x)\,dx. \]

\subsection*{Nerovnosti mezi integrály}

Nechť jsou $f$ a $g$ spojité funkce na intervalu $\langle a,b \rangle$ a nechť
platí nerovnost $f(x) \leq g(x)$ pro všechna $x\in\langle a,b \rangle$. Potom
pro jejich Riemannovy integrály platí

\[ \int_a^b f(x)\,dx \leq \int_a^b g(x)\,dx. \]

\subsection*{Newtonova formule}

Funkce $f$ je spojitá na intervalu $\langle a, b \rangle$ s primitivní funkcí
$F$, Pak platí:

\[ \int_a^b f(x) dx = F(b) - F(a) =:\left[ F(x) \right]_a^b \]

\subsection*{Per partes pro určitý integrál}

Funkce $f$ a $g$ jsou spojité na $\langle a, b \rangle$, $f$ má spojitou
derivaci na intervalu $\langle a, b \rangle$ a $G$ je primitivní funkcí k $g$
na intervalu $\langle a, b \rangle$. Potom:

\[ \int_a^b f(x)g(x) dx =  \left[ f(x)G(x) \right]_a^b - \int_a^b f'(x)G(x) dx\]

\subsection*{Věta o substituci v určitém integrálu}

Nechť pro funkce $f$ a $\varphi$ platí

\begin{enumerate}
    \item $\varphi$ a její derivace $\varphi'$ jsou spojité na $\langle \alpha,\beta \rangle$,
    \item $f$ je spojitá na $\varphi\big(\langle\alpha,\beta\rangle\big)$.
\end{enumerate}

\noindent Potom pro Riemannův integrál platí

\[ \int_\alpha^\beta f\big(\varphi(t)\big)\cdot\varphi'(t) \,\mathrm{d}t = \int_{\varphi(\alpha)}^{\varphi(\beta)}
    f(x)\,dx. \]

\subsection*{Integrace na symetrickém intervalu}

Nechť $f$ je funkce spojitá na uvažovaných intervalech.

\begin{enumerate}
    \item Je-li $f$ sudá funkce na $\langle -a,a \rangle$, pak $\displaystyle\int_{-a}^a
              f(x) dx = 2 \int_0^a f(x) dx$.
    \item Je-li $f$ lichá funkce na $\langle -a,a \rangle$, pak $\displaystyle\int_{-a}^a
              f(x) dx = 0$.
    \item Je-li $f$ periodická na $\mathbb{R}$ s periodou $T$, pak pro každé
          $a,b\in\mathbb{R}$ platí $\displaystyle\int_a^{a+T} f(x)dx = \int_b^{b+T} f(x)
              dx$.
\end{enumerate}

\subsection*{Zobecněný Riemannův integrál}

Nechť $f$ je funkce definovaná na intervalu $\langle a, b)$ pro nějaké $a \in
    \mathbb{R}$ a $b \in (a, +\infty) \cup \{+\infty\}$, která má Riemannův
integrál na intervalu $\langle a, c \rangle$ pro každé $c \in (a,b)$. Pokud
existuje konečná limita

\[ \lim_{c \to b_-} \int_a^c f(x)\, dx, \]

\noindent pak její hodnotu značíme

\[ \int_a^b f(x)\, dx, \]

\noindent nazýváme \textbf{zobecněným Riemannovým integrálem} funkce $f$ na intervalu $\langle a, b)$ a říkáme, že integrál $\int_a^b f(x)\, dx$ konverguje.

\subsection*{Absolutně konvergentní zobecněný Riemannův integrál na $\mathbb{R}$}

Buď $f$ spojitá funkce definovaná na $\mathbb{R}$. Pokud existuje konečná
limita

\[ \lim_{c\to+\infty} \int_{-c}^{c} |f(x)| \,dx, \]

\noindent pak tuto její hodnotu značíme

\[ \int_{-\infty}^{+\infty} |f(x)| \,dx \]

\noindent a o $f$ říkáme, že má \textbf{absolutně konvergentní zobecněný Riemannův integrál} na $\mathbb{R}$.
Pokud má funkce absolutně konvergentní zobecněný Riemannův integrál na $\mathbb{R}$, pak i limita

\[ \lim_{c\to+\infty} \int_{-c}^{c} f(x) \,dx \]

\noindent existuje a značíme ji

\[ \int_{-\infty}^{+\infty} f(x)\,dx. \]

\noindent Tuto hodnotu pak nazýváme \textbf{zobecněným Riemannovým integrálem $f$ na $\mathbb{R}$}

\pagebreak

\section{Funkce více proměnných}

\subsection*{Euklidovská norma a vzdálenost}

\textbf{Euklidovskou normu vektoru} $\mathbf{x} \in \mathbb{R}^n$ definujeme předpisem

\[ \| \mathbf{x} \| \equiv \sqrt{\sum_{j=1}^n x_j^2}\,. \]

\noindent \textbf{Euklidovskou vzdálenost dvou bodů} $\mathbf{x} \in \mathbb{R}^n$ a $\mathbf{y} \in \mathbb{R}^n$ pak představuje číslo

\[ d(\mathbf{x}, \mathbf{y}) \equiv \|\mathbf{x} - \mathbf{y}\| = \sqrt{\sum_{j=1}^n (x_j - y_j)^2}\,. \]

\subsection*{Okolí bodu $\mathbf{a} \in \mathbb{R}^n$}

Mějme bod $\mathbf{a} \in \mathbb{R}^n$ a poloměr $\epsilon > 0$. Potom
\textbf{okolím bodu $\mathbf{a}$ o poloměru $\epsilon$} nazýváme množinu všech
bodů $\mathbf{x} \in \mathbb{R}^n$, jejichž vzdálenost od bodu $\mathbf{a}$ je
menší než $\epsilon$ a značíme ho $U_\mathbf{a}(\epsilon)$. Tj. podrobně
rozepsáno

\[ U_\mathbf{a}(\epsilon) \equiv \big\{ \mathbf{x} \in \mathbb{R}^n \,\big|\, d(\mathbf{x}, \mathbf{a}) < \epsilon \big\} \subset \mathbb{R}^n. \]

\subsection*{Hromadný bod množiny $M \subset \mathbb{R}^n$}

Bod $\mathbf{a} \in \mathbb{R}^n$ nazýváme \textbf{hromadným bodem množiny $M
        \subset \mathbb{R}^n$}, právě když v každém okolí bodu $\mathbf{a}$ leží bod
množiny $M$ různý od $\mathbf{a}$.

\subsection*{Vnitřní bod množiny}

O bodu $\mathbf{a} \in M \subset \mathbb{R}^n$ řekneme, že je \textbf{vnitřním
    bodem množiny $M$}, právě když existuje okolí $U_{\mathbf{a}}$ bodu
$\mathbf{a}$ takové, že $U_{\mathbf{a}} \subset M$.

\subsection*{Otevřená množina}

O množině $M \subset \mathbb{R}^n$ řekneme, že je \textbf{otevřená}, právě když
pro každý bod $\mathbf{a} \in M$ existuje okolí $U_{\mathbf{a}}$ bodu
$\mathbf{a}$ takové, že $U_{\mathbf{a}} \subset M$.

\subsection*{Limita vektorové posloupnosti}

Řekneme, že posloupnost $(\mathbf{x}_k)_{k=1}^\infty$ vektorů $\mathbf{x}_k \in \mathbb{R}^n$ má \textbf{limitu} (případně \textbf{konverguje k}) $\mathbf{a} \in \mathbb{R}^n$, právě když pro každé okolí $U_\mathbf{a}$ bodu $\mathbf{a}$ existuje $N \in \mathbb{N}$ takové, že pro každé přirozené $k > N$ platí $\mathbf{x}_k \in U_\mathbf{a}$.
Tento fakt značíme $\lim\limits_{k\to\infty} \mathbf{x}_k = \mathbf{a}$.

\subsection*{Konvergence a vzdálenost}

Pro vektorovou posloupnost $(\mathbf{x}_k)_{k=1}^\infty$ platí
$\lim\limits_{k\to \infty} \mathbf{x}_k = \mathbf{a}$, právě když
$\lim\limits_{k \to \infty} \| \mathbf{x}_k - \mathbf{a} \| = 0$ (tato druhá
limita je obyčejná limita z BI-MA1).

\subsection*{Konvergence po složkách}

Uvažme posloupnost $(\mathbf{x}_k)_{k=1}^\infty$. Potom platí následující
ekvivalence: $\lim\limits_{k\to\infty} \mathbf{x}_k = \mathbf{a}$, právě když
pro každé $j\in\hat n$ platí $\lim\limits_{k \to \infty} (\mathbf{x}_k)_j =
    \mathbf{a}_j$.

\subsection*{Limita součtu a skalárního násobku posloupností}

Mějme dvě vektorové posloupnosti $(\mathbf{x}_k)_{k=1}^\infty$ a
$(\mathbf{y}_k)_{k=1}^\infty$ splňující $\lim\limits_{k\to\infty} \mathbf{x}_k
    = \mathbf{a}$ a $\lim\limits_{k\to\infty} \mathbf{y}_k = \mathbf{b}$ a $\alpha
    \in \mathbb{R}$. Potom

\begin{align}
    \lim_{k\to\infty} \big( \mathbf{x}_k + \mathbf{y}_k \big) & = \mathbf{a} + \mathbf{b}, \\
    \lim_{k\to\infty} \big( \alpha \mathbf{x}_k \big)         & = \alpha \mathbf{a}.
\end{align}

\subsection*{Limita (vektorové) funkce více proměnných}

Mějme funkci $n$ reálných proměnných $F: D_F \to \mathbb{R}^m$, $D_F \subset
    \mathbb{R}^n$, a hromadný bod $\mathbf{a}$ množiny $D_F$.

\noindent Potom \textbf{funkce $F$ má v bodě $\mathbf{a}$ limitu $\mathbf{b}\in\mathbf{R}^m$}, právě když pro každé okolí $U_\mathbf{b}$ bodu $\mathbf{b}$ existuje okolí $U_\mathbf{a}$ bodu $\mathbf{a}$ takové, že kdykoliv $\mathbf{x} \in (U_\mathbf{a} \cap D_F) \smallsetminus \{\mathbf{a}\}$ pak platí $F(\mathbf{x}) \in U_\mathbf{b}$.

\noindent Symbolicky tuto situaci zapisujeme opět jako

\[ \lim_{\mathbf{x} \to \mathbf{a}} F(\mathbf{x}) = \mathbf{b}. \]

\noindent Pokud $m = 1$, pak ještě pro $\alpha \in \{+\infty, -\infty\}$ klademe $\displaystyle\lim_{\mathbf{x} \to \mathbf{a}} F(\mathbf{x}) = \alpha$ kdykoliv

\[ (\forall U_{\alpha})(\exists U_\mathbf{a})(\forall \mathbf{x} \in \mathbb{R}^n)(x \in (U_\mathbf{a} \cap D_F) \smallsetminus \{\mathbf{a}\} \Rightarrow F(\mathbf{x}) \in U_{\alpha}). \]

\subsection*{Limita zúžení}

Mějme vektorovou funkci $F: D_F \to \mathbb{R}^m$, $D_F \subset \mathbb{R}^n$ a
hromadný bod $\mathbf{a}$ definičního oboru funkce $F$ v němž existuje limita

\[ \lim_{\mathbf{x} \to \mathbf{a}} F(\mathbf{x}) = \mathbf{b} \in \mathbb{R}^m. \]

\noindent Potom i pro $F|_M$ zúžení funkce $F$ na množinu $M$, která má $\mathbf{a}$ jako hromadný bod, platí

\[ \lim_{\mathbf{x} \to \mathbf{a}} (F|_M)(\mathbf{x}) = \mathbf{b} \in \mathbb{R}^m. \]

\subsection*{Limita vektorové funkce a limity jejích složek}

Mějme (vektorovou) funkci $F: D_F \to \R^m$, $D_F \subset \R^n$, hromadný bod
$\va \in \R^n$ množiny $D_F$ a bod $\vb \in \R^m$. Potom platí
\begin{itemize}
    \item $\displaystyle\lim_{\vx \to \va} F(\vx) = \vb$, právě když $\displaystyle \lim_{\vx \to \va} \| F(\vx) - \vb \| = 0$.
    \item Označme složky $F$ jako

          \[ F(\vx) = \big(F_1(\vx), \cdots, F_m(\vx)\big)^T, \quad x \in D_F. \]

          Pak $\lim\limits_{\vx \to \va} F(\vx) = \vb$, právě když
          $\lim\limits_{\vx\to\va} F_j(\vx) = b_j$ pro každé $j \in \hat{m}$.
\end{itemize}

\subsection*{Věta o limitě součtu, násobku}

Mějme dvě vektorové funkce $F: D_F \to \R^m$, $D_F \subset \R^n$, $G: D_G \to
    \R^m$, $D_G \subset \R^n$, bod $\va \in \R^n$, který je hromadným bodem množiny
$D_F \cap D_G$ a $\alpha \in \R$. Potom pokud existují limity
$\displaystyle\lim_{\vx\to\va} F(\vx) = \vb \in \R^m$ a
$\displaystyle\lim_{\vx\to\va} G(\vx) = \vc \in \R^m$, potom platí

\[ \lim_{\vx\to\va} \big( F(\vx) + G(\vx) \big) = \vb + \vc
    \quad \text{a} \quad
    \lim_{\vx\to\va} \alpha F(\vx) = \alpha \vb. \]

\subsection*{Věta o limitě součinu a podílu}

Mějme dvě funkce $f: D_f \to \R$, $D_f \subset \R^n$, $g: D_g \to \R$, $D_g
    \subset \R^n$, bod $\va \in \R^n$, který je hromadným bodem množiny $D_f \cap
    D_g$. Potom pokud existují limity $\displaystyle\lim_{\vx\to\va} f(\vx) = b \in
    \R$ a $\displaystyle\lim_{\vx\to\va} g(\vx) = c \in \R$, potom platí

\[ \lim_{\vx\to\va} f(\vx) \cdot g(\vx) = b \cdot c
    \quad \text{a} \quad
    \lim_{\vx\to\va} \frac{f(\vx)}{g(\vx)} = \frac{b}{c}, \ \text{pokud} \ c \neq 0. \]

\subsection*{Spojitost (vektorové) funkce}

Mějme (vektorovou) funkci $F: A \to \R^m$, $A \subset\R^n$, a bod $\va \in
    D_F$, který je hromadným bodem množiny $D_F$.

\textbf{Funkce $F$ je spojitá v bodě $\va$}, právě když

\[ \lim_{\vx \to \va} F(\vx) = F(\va). \]

\noindent Funkci $F$ nazveme \textbf{spojitou} (resp. spojitou na množině $M$), právě když je spojitá v každém bodě svého definičního oboru (resp. v každém bodě množiny $M$).

\subsection*{Spojitost součtu, násobku, součinu a podílu}

Mějme dvě vektorové funkce $F: D_F \to \R^m$, $D_F \subset \R^n$, $G: D_G \to
    \R^m$, $D_G \subset \R^n$, bod $\va \in \R^n$, který je hromadným bodem množiny
$D_F \cap D_G$ a $\alpha \in \R$. Předpokládejme, že $F$ i $G$ jsou spojité v
bodě $\va$. Potom

\begin{itemize}
    \item $F + G$ je spojitá v $\va$,
    \item $\alpha F$ je spojitá v $\va$.
\end{itemize}

\noindent Pokud je $m = 1$, pak

\begin{itemize}
    \item $F \cdot G$ je spojitá v $\va$,
    \item $\frac{F}{G}$ je spojitá v $\va$ v případě kdy $G(\va) \neq 0$.
\end{itemize}

\subsection*{$D_f$ v $k$-tém faktoru}

Buďte $f: D_f \to \R$, $D_f \subset \R$, reálná funkce jedné reálné proměnné,
$n \in \N$ a $k \in \hat{n}$. Definujme funkci ($D_f$ v $k$-tém faktoru)

\[ g(\vx) \equiv f(x_k), \ \text{pro} \ \vx \in D_g \equiv
    \underbrace{\R \times \cdots \times \R}_{k-1} \times D_f \times \underbrace{\R \times \cdots \times \R}_{n-k}. \]

Je-li $f$ spojitá, pak i $g$ je spojitá.

\subsection*{Věta o spojitosti složené (vektorové) funkce}

Mějme (vektorové) funkce $g: D_g \to \R^m$, $D_g \subset \R^n$ a $f: D_f \to
    \R^k$, $D_f \subset \R^m$. Dále předpokládejme, že $g$ je spojitá v $\va \in
    D_g$ a $f$ je spojitá a definovaná na okolí $g(\va)$. Potom je $f \circ g$
spojitá v bodě $\va$.

\subsection*{Parciální derivace (v bodě)}

Mějme reálnou funkci $n$ reálných proměnných $f: D_f \to \R$, $D_f \subset
    \R^n$, definovanou na okolí bodu $\va \in D_f$ a $j \in \hat n$.

Existuje-li limita

\[ \lim_{h \to 0} \frac{f(\va + h \ve_j) - f(\va)}{h}, \]

pak její hodnotu nazýváme \textbf{parciální derivací funkce $f$ v bodě $\va$
    podle $j$-té proměnné} a značíme ji $\frac{\partial f}{\partial x_j}(\va)$, případně $\partial_{x_j} f(\va)$.

Označme $M$ jako množinu všech vnitřních bodů $\va$ množiny $D_f$, v kterých
existuje limita (předchozí limita). Potom funkci přiřazující hodnotu $\frac{\partial{f}}{\partial x_j}(\va)$ každému $\va \in M\subset\R^n$ nazýváme \textbf{parciální
    derivací funkce $f$ podle $j$-té proměnné} a značíme ji

\[ \frac{\partial f}{\partial x_j}, \quad \text{případně} \quad \partial_{x_j} f. \]

\subsection*{Gradient}

Mějme reálnou funkci $n$ reálných proměnných $f: D_f \to \R$, $D_f \subset
    \R^n$ mající všechny parciální derivace v bodě $\va \in D_f$. Potom řádkový
vektor

\[ \left( \frac{\partial f}{\partial x_1}(\va), \frac{\partial f}{\partial x_2}(\va), \ldots, \frac{\partial f}{\partial x_n}(\va) \right) \in \R^{1,n} \]

\noindent nazýváme \textbf{gradientem funkce $f$ v bodě $\va$} a používáme pro něj značení

\[ \nabla f(\va) \quad \text{nebo} \quad \mathrm{grad} f(\va). \]

\subsection*{Derivace (vektorové) funkce}

Mějme zobrazení $F: D_F \to \R^m$, $D_F \subset \R^n$, definované na okolí bodu
$\va$.

Derivací zobrazení $F$ v bodě $\va$ nazýváme matici $DF(\va) \in \R^{m,n}$
splňující

\[ \lim_{\vx \to \va} \frac{\|F(\vx) - F(\va) - DF(\va) \cdot (\vx - \va) \|}{\|\vx - \va\|} = 0. \]

\subsection*{Složky matice $DF(\va)$ a její jednoznačnost}

Pokud má zobrazení $F: D_f \to \R^m$, $D_f \subset \R^n$, definované na okolí
bodu $\va$, derivaci $DF(\va) \in \R^{m,n}$ v bodě $\va$, potom

\[ DF(\va) = \begin{pmatrix}
        \frac{\partial F_1}{\partial x_1}(\va) & \frac{\partial F_1}{\partial x_2}(\va) & \cdots & \frac{\partial F_1}{\partial x_n}(\va) \\ \frac{\partial F_2}{\partial x_1}(\va) & \frac{\partial F_2}{\partial x_2}(\va) & \cdots & \frac{\partial F_2}{\partial x_n}(\va) \\ \vdots & \vdots & \ddots & \vdots \\ \frac{\partial F_m}{\partial x_1}(\va) & \frac{\partial F_m}{\partial x_2}(\va) & \cdots & \frac{\partial F_m}{\partial x_n}(\va)
    \end{pmatrix}. \]

\noindent Odtud ihned také plyne, že je tato matice dána jednoznačně, existuje-li.

\subsection*{Hessova matice}

Na derivaci, resp. gradient, funkce $f: D_f \to \R$, $D_f \subset \R^n$, lze
nahlížet jako na zobrazení $Df: A \to \R^n$, $A \subset D_f$, jeho derivací v
bodě $\va \in A$ je pak matice typu $\R^{n,n}$, kterou nazýváme
\textbf{Hessovou maticí} a značíme $\nabla^2 f(\va)$. Pokud existuje, pak platí

\[ \nabla^2 f(\va) = \begin{pmatrix}
        \frac{\partial^2 f}{\partial x_1^2}(\va) & \frac{\partial^2 f}{\partial x_2 \partial x_1}(\va) & \cdots & \frac{\partial^2 f}{\partial x_n \partial x_1}(\va) \\ \frac{\partial^2 f}{\partial x_1 \partial x_2}(\va) & \frac{\partial^2 f}{\partial x_2^2}(\va) & \cdots & \frac{\partial^2 f}{\partial x_n \partial x_2}(\va) \\ \vdots & \vdots & \cdots & \vdots \\ \frac{\partial^2 f}{\partial x_1 \partial x_n}(\va) & \frac{\partial^2 f}{\partial x_2 \partial x_n}(\va) & \cdots & \frac{\partial^2 f}{\partial x_n^2}(\va)
    \end{pmatrix}. \]

\subsection*{Derivace složené funkce}

Mějme zobrazení $F: D_F \to \R^k$, $D_F \subset \R^m$ a $G: D_G \to \R^m$, $D_G
    \subset \R^n$ a bod $\va \in D_G$ takové, že existují $DG(\va)$ a $D F\big(
    G(\va) \big)$. Potom existuje i derivace složeného zobrazení $F \circ G$ v bodě
$\va$ a platí

\[ D\big( F \circ G \big)(\va) = DF\big(G(\va)\big) \cdot DG(\va). \]

\subsection*{Derivace ve směru}

Nechť $f: D_f \to \R$, $D_f \subset \R^n$ má derivaci v bodě $\va \in D_f$. Buď
$\vv$ vektor délky $1$.

\noindent Potom existuje limita (tzv. \textbf{derivace funkce $f$ ve směru $\vv$ v bodě $\va$})

\[ \frac{\partial f}{\partial \vv}(\va) \equiv \partial_{\vv} f(\va) \equiv \lim_{h \to 0} \frac{f(\va + h\vv) - f(\va)}{h} \]

\noindent a je rovna $\langle \nabla f(\va)^T \mid \vv \rangle$.

\pagebreak
\section{Asymptotika}

\subsection*{Asymptotické horní meze $o$ a $\mathcal{O}$}

\begin{align*}
    a_n & = \mathcal{O}(b_n) &  & \stackrel{\text{def}}{\Leftrightarrow} &  & (\exists c > 0)(\exists N \in \N)(\forall n \in \N)\big(n > N \Rightarrow |a_n| \leq c |b_n|\big), \\
    a_n & = o(b_n)           &  & \stackrel{\text{def}}{\Leftrightarrow} &  & (\forall c > 0)(\exists N \in \N)(\forall n \in \N)\big(n > N \Rightarrow |a_n| < c |b_n|\big).
\end{align*}

\subsection*{Dolní asymptotická mez $\Omega$}

Mějme dvě posloupnosti $(a_n)_{n=1}^\infty$ a $(b_n)_{n=1}^\infty$. Řekneme, že
\textbf{posloupnost $(a_n)_{n=1}^\infty$ je asymptoticky zdola omezená
    posloupností $(b_n)_{n=1}^\infty$}, symbolicky $a_n = \Omega(b_n)$ pro $n \to
    \infty$, právě když existuje kladná konstanta $c \in \R$ a přirozené $N \in \N$
tak, že pro všechna $n \geq N$ platí

\[ |a_n| \geq c \cdot |b_n|. \]

\noindent Vlastnosti:

\begin{itemize}
    \item $a_n = \Omega(b_n)$, právě když $b_n = \mathcal{O}(a_n)$.
    \item $a_n = \Omega(a_n)$.
    \item Vztah $\Omega$ je tranzitivní.
\end{itemize}

\subsection*{Dolní striktní asymptotická mez $\omega$}

Mějme dvě posloupnosti $(a_n)_{n=1}^\infty$ a $(b_n)_{n=1}^\infty$. Řekneme, že
\textbf{posloupnost $(a_n)_{n=1}^\infty$ je asymptoticky zdola striktně omezená
    posloupností $(b_n)_{n=1}^\infty$}, symbolicky $a_n = \omega(b_n)$ pro $n \to
    \infty$, právě když pro každé kladné $c \in \R$ existuje $N \in \N$ tak, že pro
všechna $n \geq N$ platí

\[ |a_n| > c \cdot |b_n|. \]

\noindent Vlastnosti:

\begin{itemize}
    \item  $a_n = \omega(b_n)$, právě když $b_n = o(a_n)$.
    \item Pokud $a_n = \omega(b_n)$, pak $a_n = \Omega(b_n)$.
    \item $\omega$ je tranzitivní.
\end{itemize}

\subsection*{Asymptotická těsná mez $\Theta$}

Mějme dvě posloupnosti $(a_n)_{n=1}^\infty$ a $(b_n)_{n=1}^\infty$. Řekneme, že
\textbf{posloupnost $(a_n)_{n=1}^\infty$ je téhož řádu jako posloupnost
    $(b_n)_{n=1}^\infty$}, symbolicky $a_n = \Theta(b_n)$ pro $n \to \infty$, právě
když existují kladné konstanty $c_1,c_2 \in \R$ a $N \in \N$ tak, že pro
všechna $n \geq N$ platí

\[ c_1 |b_n| \leq |a_n| \leq c_2 |b_n|. \]

\noindent Vlastnosti:

\begin{itemize}
    \item $a_n = \Theta(b_n)$, právě když $b_n = \Theta(a_n)$.
    \item Vztah $\Theta$ kombinuje $\mathcal{O}$ a $\Omega$ v následujícím smyslu: $a_n =
              \Theta(b_n)$, právě když $a_n = \Omega(b_n)$ a $a_n = \mathcal{O}(b_n)$.
    \item $\Theta$ je tranzitivní.
\end{itemize}

\pagebreak


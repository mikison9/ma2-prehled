\section{Důkazy}

\subsection*{Důkaz věty o poloměru konvergence}

Pro libovolné $x\in\R$ různé od $c$ dostáváme

\[ \lim_{k\to\infty} \bigg| \frac{a_{k+1} (x-c)^{k+1}}{a_k (x-c)^k} \bigg| = \lim_{k\to\infty} |x - c| \cdot \bigg| \frac{a_{k+1}}{a_k} \bigg| = |x - c| \cdot L. \]

\noindent Shrnujeme, že pokud
\begin{itemize}
    \item $|x - c| \cdot L < 1$, tedy $|x - c| < R$, pak podle d'Alembertova kritéria zkoumaná řada konverguje absolutně,
    \item $|x - c| \cdot L > 1$, tedy $|x - c| > R$, pak podle podílového kritéria je $\displaystyle\lim_{k\to\infty} |a_k (x-c)^k| = +\infty$. Tudíž nemůže být splněna nutná podmínka konvergence zkoumané řady (tj. neplatí $\displaystyle\lim_{k\to\infty} a_k (x-c)^k = 0$).
\end{itemize}

\subsection*{Důkaz per partes}

Tvrzení věty můžeme přímo ověřit derivováním. Platí

\[ \left( fG - \int f' G \right)' = (fG)' - f'G = f'G + fG' - f'G = fG' = fg. \]

\noindent Ve výpočtu jsme dále použili známého Leibnizova pravidla pro derivování součinu
funkcí.

\pagebreak

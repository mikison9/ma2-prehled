\section{Důkazy}

\subsection*{Důkaz věty o poloměru konvergence}

Pro libovolné $x\in\R$ různé od $c$ dostáváme

\[ \lim_{k\to\infty} \bigg| \frac{a_{k+1} (x-c)^{k+1}}{a_k (x-c)^k} \bigg| = \lim_{k\to\infty} |x - c| \cdot \bigg| \frac{a_{k+1}}{a_k} \bigg| = |x - c| \cdot L. \]

\noindent Shrnujeme, že pokud
\begin{itemize}
    \item $|x - c| \cdot L < 1$, tedy $|x - c| < R$, pak podle d'Alembertova kritéria zkoumaná řada konverguje absolutně,
    \item $|x - c| \cdot L > 1$, tedy $|x - c| > R$, pak podle podílového kritéria je $\displaystyle\lim_{k\to\infty} |a_k (x-c)^k| = +\infty$. Tudíž nemůže být splněna nutná podmínka konvergence zkoumané řady (tj. neplatí $\displaystyle\lim_{k\to\infty} a_k (x-c)^k = 0$).
\end{itemize}

\subsection*{Důkaz per partes v neurčitém integrálu}

Tvrzení věty můžeme přímo ověřit derivováním. Platí

\[ \left( fG - \int f' G \right)' = (fG)' - f'G = f'G + fG' - f'G = fG' = fg. \]

\noindent Ve výpočtu jsme dále použili známého Leibnizova pravidla pro derivování součinu
funkcí.

\subsection*{Důkaz limity součtu vektorových posloupností}

Předpoklad: $\lim\limits_{k\to\infty} \vx_k = \va \wedge
    \lim\limits_{k\to\infty} \vy_k = \vb$, z toho plyne: $\forall j \in \hat{n}:
    \lim\limits_{k\to\infty} (\vx_k)_j = \va_j \wedge \lim\limits_{k\to\infty}
    (\vy_k)_j = \vb_j$ (Konvergence po složkách).

\vspace{1em}

\noindent Chceme dokázat, že $\lim\limits_{k\to\infty} (\vx_k + \vy_k) = \va + \vb$.

\[ \forall j \in \hat{n}: \va_j + \vb_j = \lim\limits_{k\to\infty} (\vx_k)_j + \lim\limits_{k\to\infty} (\vy_k)_j = \lim\limits_{k\to\infty} (\vx_k)_j + (\vy_k)_j\]

\noindent Poslední krok jsme mohli udělat díky MA1 (limita součtu jednoduchých posloupností).

\noindent Díky větě o konvergenci po složkách víme, že posloupnost konverguje k bodu \textbf{právě tehdy}, když konvergují její složky.

\subsection*{Důkaz, že $|R_{n,a}(x)| \leq |R_{n-1,a}(x)|$}

Z Taylorova vzorce: \[ |R_{n,a}(x)| = |f(x) - T_{n, a}(x)| \hspace{3mm}\text{a}\hspace{3mm} |R_{n-1,a}(x)| = |f(x) - T_{n - 1, a}(x)| \]

\noindent Protože $T_{n - 1, a}(x)$ je polynom stupně nejvýše $n$, různý od $T_{n, a}(x)$, pak platí (Věta o nejlepší aproximaci):

\[ |f(x) - T_{n, a}(x)| \leq |f(x) - T_{n - 1, a}(x)| \]

\noindent Což je to samé jako:

\[ |R_{n,a}(x)| \leq |R_{n-1,a}(x)| \]

\subsection*{Důkaz věty o substituci I v neurčitém integrálu}

$F$ je primitivní funkcí k funkci $f$, tj. $F'(x) = f(x)$ pro každé $x\in(a,b)$. Podle věty o derivaci složené funkce dostaneme

\[ (F\circ \varphi)'(x) = F'\big( \varphi(x)  \big) \cdot \varphi'(x) = f\big( \varphi(x) \big) \cdot \varphi'(x), \]

\noindent pro každé $x\in(\alpha,\beta)$.

\subsection*{Důkaz, že $\forall x \in \langle a, b\rangle:  f(x) \leq g(x) \Rightarrow \int_a^b f(x)dx \leq \int_a^b g(x)dx$}

Nejprve určitě platí pro libovolné $m, n \in \langle a, b \rangle$, kde $n >
    m$:

\[ \sup \{f(x)| x \in \langle m, n \rangle\} \leq \sup \{g(x)| x \in \langle m, n \rangle\}\]

\noindent Pak vezměme inegrální součet obou funkcí:

\[ \mathcal{J}(\sigma,f) = \sum_{i=1}^n f(\alpha_i) \Delta_i, \]

\noindent kde $\alpha_i$ budeme brát vždy jako $f(\alpha_i) = \sup \{f(x)| x \in \langle x_{i-1}, x_{i} \rangle\}$ (argument pro které nabývá funkce na tomto intervalu maximum).

\noindent Potom platí $\forall \sigma$ dělení intervalu $\langle a, b\rangle$:

\begin{align*}
    \mathcal{J}(\sigma, f)                            & \leq \mathcal{J}(\sigma, g)                            \\
    \lim\limits_{n\to\infty} \mathcal{J}(\sigma_n, f) & \leq \lim\limits_{n\to\infty} \mathcal{J}(\sigma_n, g) \\
    \int_a^b f(x)dx                                   & \leq \int_a^b g(x)dx
\end{align*}

\subsection*{Důkaz, že $f(x, y) = x^2 + y^2$ má v bodě $\theta$ ostré lokální minimum}

Chceme dokázat, že:

\[ (\exists U_\theta(\epsilon))(\forall \vx \in U_\theta \cap D_f)(\vx \ne \theta \Rightarrow f(\vx) > f(\theta)) \]

\noindent Pro $x \ne 0$ a $y \ne 0$:

\[ x^2 + y^2 > 0 \]

\noindent platí vždy $\Rightarrow$ můžeme vzít libovolné $\epsilon$.

\subsection*{Důkaz, že $f(x, y) = x^2 - y^2$ nemá v bodě $\theta$ extrém.}

$f(x, y) = x^2 - y^2$, $\frac{\partial f}{\partial x}(x, y) = 2x$, $\frac{\partial f}{\partial y}(x, y) = -2y$

\noindent $\nabla f(x, y) = (2x, -2y)$, $\nabla f(0, 0) = \theta$

\vspace{0.5em}

\noindent Chceme dokázat:

\[ (\forall U_\theta(\epsilon))(\exists \vx_1, \vx_2 \in U_a \cap D_f)(f(\vx_1) > f(\theta) \wedge f(\vx_2) < f(\theta)) \]

\noindent Pro $\vx_1, \vx_2$ má platit: $d(\vx, \theta) < \epsilon \Leftrightarrow \sqrt{x^2 + y^2} < \epsilon$.
Pro $\vx_1$ zafixujeme $y = 0$ a pro $\vx_2$ zafixujeme $x = 0$. Potom pro:

\[ \vx_1: x < \epsilon \hspace{3mm} \text{a} \hspace{3mm} \vx_2: y < \epsilon \]

\noindent Pokud vybereme pro libovolné okolí $\vx_1$ a $\vx_2$ podle těchto podmínek, tak budou patřit do ukolí $U_\theta$ a bude pro ně splněna podmínka.

\subsection*{Důkaz věty o konvergenci a vzálenosti}

Chceme dokázat: $\lim\limits_{k\to\infty} \vx_k = \va \Leftrightarrow
    \lim\limits_{k\to\infty} \| \vx_k - a \| = 0$

\noindent Ekvivalenci dokážeme tím, že dokážeme $A \Rightarrow B$ a $B \Rightarrow A$

\begin{enumerate}
    \item $\Rightarrow$:
          Předpoklad: $(\forall U_a(\epsilon_0))(\exists n_0 \in \mathbb{N})(\forall k > n_0: \vx_k \in U_a)$, tím pádem:

          \[ \sqrt{({(\vx_k)}_1 - \va_1)^2 + \cdots + ({(\vx_k)}_j - \va_j)^2} < \epsilon_0 \]
          \noindent Chceme dokázat, že: $(\forall U_0(\epsilon_1))(\exists n_1 \in \mathbb{N})(\forall k > n_1: \| \vx_k - \va \| \in U_0)$, tím pádem:

          \[ \sqrt{({(\vx_k)}_1 - \va_1)^2 + \cdots + ({(\vx_k)}_j - \va_j)^2} < \epsilon_1 \]

          \noindent Od nějakého $n_1$. Je to stejná rovnost jako u předpokladu a ten nám garantuje, že pro libovolná $U_a$ existuje $n_0$, které tuto nerovnost splňuje. Tím pádem stačí vzít $U_a(\epsilon_1)$ a k němui správné $n_0$ a pak položit $n_1 = n_0$.
    \item $\Leftarrow$: úplně stejně
\end{enumerate}

\subsection*{Důkaz principu superpozice}

\begin{align*}
    x_{n + k} + \sum_{i = 0}^{k - 1} c_{i, n}x_{n + i} & = b_n       \\
    x_{n + k} + \sum_{i = 0}^{k - 1} c_{i, n}x_{n + i} & = \hat{b}_n
\end{align*}

\noindent $(X_n)_{n = n_0}^\infty$ a $(Y_n)_{n = n_0}$ jsou řešením těchto dvou rovnic. Potom $(X_n + \alpha Y_n)_{n = n_0}^\infty$ je řešením:

\[ x_{n + k} + \sum_{i = 0}^{k - 1} c_{i, n}x_{n + i} = b_n + \alpha \hat{b}_n\]

\noindent Dosazením do prvních dvou rovnic dostaneme:

\begin{align*}
    X_{n + k} + \sum_{i = 0}^{k - 1} c_{i, n}X_{n + i} & = b_n       \\
    Y_{n + k} + \sum_{i = 0}^{k - 1} c_{i, n}Y_{n + i} & = \hat{b}_n
\end{align*}

Dosazením do poslední rovnice dostaneme:
\begin{center}
    $X_{n + k} + \alpha Y_{n + k} + \sum_{i = 0}^{k - 1} c_{i, n}X_{n + i} + c_{i, n} \alpha Y_{n + i} = X_{n + k} + \sum_{i = 0}^{k - 1} (c_{i, n}X_{n + i}) +  \alpha (Y_{n + k} + \sum_{i = 0}^{k - 1} c_{i, n} Y_{n + i}) = b_n + \alpha \hat{b}_n$
\end{center}

\subsection*{Důkaz, že $\forall x \in \mathbb{R}: e^x > 0$}

\begin{enumerate}
    \item $e^0 = \sum_{k=0}^\infty \frac{0^k}{k!} = 1 + 0 + 0 + \cdots = 1.$
    \item \begin{align*}
              e^x \cdot e^y & = \left( \sum_{k=0}^\infty \frac{x^k}{k!} \right) \cdot \left( \sum_{\ell = 0}^\infty \frac{y^\ell}{\ell!} \right) =                                                         \\
                            & = \sum_{k=0}^\infty \sum_{\ell=0}^k \frac{x^\ell}{\ell!} \frac{y^{k-\ell}}{(k-\ell)!} = \sum_{k=0}^\infty \frac{1}{k!} \sum_{\ell = 0}^k \binom{k}{\ell} x^\ell y^{k-\ell} = \\
                            & = \sum_{k=0}^\infty \frac{(x+y)^k}{k!} = e^{x+y}.
          \end{align*}
    \item $\forall x \in \mathbb{R}$ platí tím pádem: $e^xe^{-x} = 1$, tím pádem $\forall x \in \mathbb{R}:  e^x \ne 0$.
\end{enumerate}

\pagebreak
